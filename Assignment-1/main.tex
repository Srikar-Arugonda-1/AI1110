\documentclass[12pt, journal]{IEEEtran}

\usepackage{tfrupee}
\usepackage{enumitem}
\usepackage{amsmath}
\usepackage{amssymb}

\title{Assignment-1 \\ \Large AI1110: Probability and Random Variables \\ \large Indian Institute of Technology Hyderabad}
\author{Arugonda Srikar \\ \normalsize CS22BTECH11008}

\begin{document}

	\maketitle

	\textbf{Solution}

	\begin{enumerate}[label=(\roman*)]
		\item 
			E denotes the event that the card drawn is spade 
			\begin{align}
				P(E) = \dfrac{13}{52} = \dfrac{1}{4}
			\end{align}
			F denotes the event that card drawn is ace 
			\begin{align}
				P(F) = \dfrac{4}{52} = \dfrac{1}{13} \\
				P(E \cap F) = \dfrac{1}{52}\\
				P(E).P(F) = \dfrac{1}{4} \times \dfrac{1}{13} = \dfrac{1}{52}
			\end{align}
			\begin{align}
				\therefore~P(E \cap F) = P(E).P(F)
			\end{align}
			$\therefore$  E and F are independent events. \\
		\item
			E denotes the event that the card drawn is black 
			\begin{align}
				P(E) = \dfrac{26}{52} = \dfrac{1}{2}
			\end{align}
			F denotes the event that card drawn is a king 
			\begin{align}
				P(F) = \dfrac{4}{52} = \dfrac{1}{13} \\
				P(E \cap F) = \dfrac{2}{52} = \dfrac{1}{26} \\
				P(E).P(F) = \dfrac{1}{2} \times \dfrac{1}{13} = \dfrac{1}{26}
			\end{align}
			\begin{align}
				\therefore~P(E \cap F) = P(E).P(F)
			\end{align}
			$\therefore$  E and F are independent events. \\
		\item
			E denotes the event that the card drawn is king or queen
			\begin{align}
				P(E) = \dfrac{8}{52} = \dfrac{2}{13}
			\end{align}
			F denotes the event that card drawn is a queen or jack 
			\begin{align}
				P(F) = \dfrac{8}{52} = \dfrac{2}{13} \\
				P(E \cap F) = \dfrac{4}{52} = \dfrac{1}{13}\\
				P(E).P(F) = \dfrac{2}{13} \times \dfrac{2}{13} = \dfrac{4}{169}
			\end{align}
			\begin{align}
				\therefore~P(E \cap F) \neq P(E).P(F)
			\end{align}
			$\therefore$  E and F are not independent events. \\
		
	\end{enumerate}
\end{document}
